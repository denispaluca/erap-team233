\section{Zusammenfassung und Ausblick}
Das Entropiefunktion wurde mit unterschiedlichen Algorithmen erfolgreich implementiert und mit einer Referenzimplementierung verglichen. Die Verwendung von Approximationen für die Logarithmusfunktion hat es uns leichter gemacht, vektorisierte Versionen unsere Algorithmen zu implementieren.

Wenn wir die Entropie von $n$ Zufallszahlen berechnen, die mit \emph{rand} und \emph{/dev/urandom} generiert sind, erhalten wir eine Entropie, die fast (aber nicht exakt) gleich $\log_2(n)$ ist. Das deutet darauf, dass die Zufallsgeneratoren gut aber nicht perfekt sind.

Der Einsatz von SSE-Registern und SIMD-Befehle hat für eine deutliche Performance-Steigerung gesorgt.  Was wir nicht erwartet hatten, war die Verbesserung der Genauigkeit durch die Kahans-Summe. Wir haben auch festgestellt, dass die Leistung der Assembler-Implementierung in vielen fällen deutlich besser als der C-Implementierung war.

Zusammenfassend lässt sich sagen, dass wir ASM-SIMD-DEG4 Implementierung empfehlen, da es die genaueste nach log2f und leistungsfähigste ist.

\newpage